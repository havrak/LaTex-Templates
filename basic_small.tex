%%% Tento soubor obsahuje základní konfiguraci používanou napříč dokumenty

%%% Lepší užití A5 stránky, standardně latex dost plýtvá
\addtolength{\topmargin}{-1.6in}
\setlength\textheight{200mm}
\setlength\textwidth{138mm}
\setlength\oddsidemargin{-20mm}
\setlength\evensidemargin{-20mm}

%%% Běžně používané balíčky
\usepackage[utf8]{inputenc}
\usepackage[czech]{babel}
\usepackage[T1]{fontenc}
\usepackage{lmodern,textcomp}

\usepackage[a-2u]{pdfx}         % metadata v pdf
\usepackage{graphicx}						% vkládání obrázků
\usepackage{caption}						%	popisky
\usepackage{hyperref} 					% odstranění červených okrajů v obsahu
\usepackage{tabularx}           % dynamické tabulkové prostřední
\usepackage{xcolor,colortbl}    % větší barevné množnosti
\usepackage{textpos}            % větší kontrola nad pozicí textu
\usepackage{amsmath}						% rozšíření pro sazbu matematiky
\usepackage{amsthm}							% sazba vět, definic apod.
\usepackage{amssymb} 						% větší kolekce symbolů do matematiky
\usepackage{longtable}          % tabulka s podporou zalamování
\usepackage{fancyhdr}						% možnost stylizovat záhlaví
\usepackage{xurl}								% umožní url zalomit všude
\usepackage{enumitem}           % rozšíření způsobů indexování listu
\usepackage{multicol}           % prostředí pro psaní textu ve sloupcích
\usepackage{pgfplots}           % grafy
\usepackage{upgreek} 						% pro sazbu řeckých písmen
\usepackage{fixmath}            % zpravuje zápis v matematickém módu dle ISO norem
\usepackage{bbm} 							  % další matematické symboly
\usepackage{esint} 						  % další integrály

%%% Základní nastavení balíčků
\def\columnseprulecolor{\color{black}}
\setlength{\columnseprule}{0.3pt}

\hypersetup{pdfborder=0 0 0}
