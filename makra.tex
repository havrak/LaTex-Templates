%%% Tento soubor obsahuje definice různých užitečných maker a prostředí %%%

%%% Drobné úpravy stylu

% Tato makra přesvědčují mírně ošklivým trikem LaTeX, aby hlavičky kapitol
% sázel příčetněji a nevynechával nad nimi spoustu místa. Směle ignorujte.
\makeatletter
\def\@makechapterhead#1{
  {\parindent \z@ \raggedright \normalfont
   \Huge\bfseries \thechapter. #1
   \par\nobreak
   \vskip 20\p@
}}
\def\@makeschapterhead#1{
  {\parindent \z@ \raggedright \normalfont
   \Huge\bfseries #1
   \par\nobreak
   \vskip 20\p@
}}
\makeatother

\renewcommand{\chaptermark}[1]{%
  \markboth{#1}{}}

% Toto makro definuje kapitolu, která není očíslovaná, ale je uvedena v obsahu.
\def\chapwithtoc#1{
\chapter*{#1}
\addcontentsline{toc}{chapter}{#1}
}

% Trochu volnější nastavení dělení slov, než je default.
\lefthyphenmin=2
\righthyphenmin=2

% Zapne černé "slimáky" na koncích řádků, které přetekly, abychom si
% jich lépe všimli.
% \overfullrule=1mm

% Definuje prázdnou stranu stranu

\newcommand\blankpage{
\newpage

\begin{center}
\vspace*{\fill}
  {Prázdná strana}
\vspace*{\fill}
\end{center}

}

%%% Makra pro definice, věty, tvrzení, příklady, ... (vyžaduje baliček amsthm)
\makeatletter
\def\th@plain{%
  \thm@notefont{}% same as heading font
  \itshape % body font
}
\def\th@definition{%
  \thm@notefont{}% same as heading font
  \normalfont % body font
}
\makeatother

\theoremstyle{plain}
\newtheorem{veta}{Věta}
\newtheorem{lemma}[veta]{Lemma}
\newtheorem{tvrz}[veta]{Tvrzení}
\newtheorem{definice}{Definice}

\theoremstyle{remark}
\newtheorem*{dusl}{Důsledek}
\newtheorem*{pozn}{Poznámka}
\newtheorem*{prikl}{Příklad}

%%% Prostředí pro důkazy

\newenvironment{dukaz}{
  \par\medskip\noindent
  \textit{Důkaz}.
}{

\rightline{$\square$}
}

%%% Vychytávky pro tabulky
\newcommand{\pulrad}[1]{\raisebox{1.5ex}[0pt]{#1}}
\newcommand{\mc}[1]{\multicolumn{1}{c}{#1}}

%%% Prostor reálných, resp. přirozených čísel
\DeclareMathOperator{\R}{\mathbb{R}}
\DeclareMathOperator{\N}{\mathbb{N}}
\DeclareMathOperator{\Q}{\mathbb{Q}}
\DeclareMathOperator{\C}{\mathbb{C}}
\DeclareMathOperator{\F}{\mathbb{F}}
\DeclareMathOperator{\Z}{\mathbb{Z}}
\DeclareMathOperator{\ED}{\mathbb{E}}

\DeclareMathOperator{\mgrad}{\text{grad}\,}
\DeclareMathOperator{\coord}{\text{coord}}
\DeclareMathOperator{\mdiv}{\mathrm{div}\,}
\DeclareMathOperator{\mrot}{\mathrm{rot}\,}

%%% Zkratka pro komentáře
\newcommand{\lcom}{\left\langle\left\langle} %% decrepated
\newcommand{\rcom}{\right\rangle\right\rangle} %% decrepated
\newcommand{\com}[1]{\left\langle\left\langle #1 \right\rangle\right\rangle}

%%% Rovná se s indexy
\DeclareMathOperator{\eqlh}{\mathrel{\stackrel{\makebox[0pt]{\mbox{\normalfont\tiny L'H}}}{=}}}
\DeclareMathOperator{\eqpp}{\mathrel{\stackrel{\makebox[0pt]{\mbox{\normalfont\tiny PP}}}{=}}}
\newcommand{\eqi}[1]{\mathrel{\stackrel{\makebox[0pt]{\mbox{\normalfont\tiny #1}}}{=}}}

\newcommand*{\ucheck}[1]{\underaccent{\check}{#1}}
\newcommand*{\uwidecheck}[1]{\underaccent{\widecheck{\hphantom{#1}}}{#1}}

%%% Čára pro dvojité podtržení
\def\doubleunderline#1{\underline{\underline{#1}}}\makeatletter

%%% Prostření tikzcd pro slovanské jazyky s ošetřenou uvozovkou
\newenvironment{tikzcdi}{\shorthandoff{"}\begin{tikzcd}}{\end{tikzcd}\shorthandon{"}}%

%%% Užitečné operátory pro statistiku a pravděpodobnost
\DeclareMathOperator{\pr}{\textsf{P}}
\DeclareMathOperator{\E}{\textsf{E}\,}
\DeclareMathOperator{\var}{\textrm{var}}
\DeclareMathOperator{\sd}{\textrm{sd}}

%%% Příkaz pro transpozici vektoru/matice
\newcommand{\T}[1]{#1^\top}

%%% Vychytávky pro matematiku
\newcommand{\goto}{\rightarrow}
\newcommand{\gotop}{\stackrel{P}{\longrightarrow}}
\newcommand{\maon}[1]{o(n^{#1})}
\newcommand{\abs}[1]{\left|{#1}\right|}
\newcommand{\dint}{\int_0^\tau\!\!\int_0^\tau}
\newcommand{\isqr}[1]{\frac{1}{\sqrt{#1}}}

